% compile this file with xelatex
\documentclass[12pt]{article}
\usepackage{graphicx}
\usepackage[margin=2cm, a4paper]{geometry}
\usepackage{setspace}
\usepackage{pdfpages}
\usepackage{float}
\usepackage{ctex}
\usepackage{amsmath}
\usepackage{fancyvrb}
\usepackage{amssymb}
\usepackage{minted}
\usepackage{enumitem}
\usepackage[colorlinks,linkcolor=blue]{hyperref}

\usepackage{xeCJK}
\setCJKmainfont{Noto Sans TC}

\renewcommand{\contentsname}{Contents}
\renewcommand{\figurename}{Figure}
\renewcommand{\tablename}{Table}
\hypersetup{
    colorlinks=true,
    linkcolor=black,
    filecolor=magenta,      
    urlcolor=blue,
}
\newcommand{\mytitle}{數位影像處理 Homework Assignment \#2}
\newcommand{\myauthor}{B13902022 賴昱錡}

\usepackage{fancyhdr}
\pagestyle{fancy}
\fancyhead{}
\fancyhead[L]{\mytitle}
\fancyhead[R]{\myauthor}

\title{\mytitle}
\author{\myauthor}
\date{}

\begin{document}

\onehalfspacing
\maketitle

\section{Part A}

\newpage
\section{Part B}
\subsection{Source Code}
\subsubsection{Usage}
\begin{Verbatim}[linenos]
python3 HistEqualization.py [*.png or *.jpg]
\end{Verbatim}

One example is \texttt{python3 HistEqualization.py monica.png}, note that the program takes one \textbf{png} or \textbf{jpg} file as the only argument
\begin{minted}[frame=lines,framesep=2mm,baselinestretch=1.2,linenos,breaklines]{python}
import cv2 as cv
import numpy as np
import sys
import matplotlib.pyplot as plt

assert len(sys.argv) == 2, "The program takes one file as argument!"
assert sys.argv[1].endswith('jpg') or sys.argv[1].endswith('png'), "This is not image file!"

filename = ""

try:
    filename = sys.argv[1]
except:
    print('Please enter filename!')

img = cv.imread(filename)
grey_img = cv.cvtColor(img, cv.COLOR_BGR2GRAY)
equalized_img = cv.equalizeHist(grey_img)

plt.subplot(2, 2, 1)
plt.hist(grey_img.ravel(),256,[0,256]);
plt.ylabel('count of pixels')
plt.xlabel('intensity')
plt.title('Original Greyscale Image')

plt.subplot(2, 2, 2)
plt.hist(equalized_img.ravel(),256,[0,256]);
plt.ylabel('count of pixels')
plt.xlabel('intensity')
plt.title('Histogram Equalized Image')

grey_img = cv.cvtColor(grey_img, cv.COLOR_GRAY2RGB)
plt.subplot(2, 2, 3)
plt.imshow(grey_img)
plt.axis('off')

equalized_img = cv.cvtColor(equalized_img, cv.COLOR_GRAY2RGB)
plt.subplot(2, 2, 4)
plt.imshow(equalized_img)
plt.axis('off')

plt.tight_layout()
plt.show()
\end{minted}
\subsection{Explanation}
\subsection{Demo}


\end{document}
% how to display codes?
% \begin{minted}[frame=lines,framesep=2mm,baselinestretch=1.2,linenos,breaklines]{python}
% \end{minted}

% how to display images?
% \begin{figure}[H]
%     \centering
%     \includegraphics[width=0.5\linewidth]{}
%     \caption{Caption}
% \end{figure}
% test