% compile this file with xelatex
\documentclass[12pt]{article}
\usepackage{graphicx}
\usepackage[margin=2cm, a4paper]{geometry}
\usepackage{setspace}
\usepackage{pdfpages}
\usepackage{float}
\usepackage{ctex}
\usepackage{amsmath}
\usepackage{fancyvrb}
\usepackage{amssymb}
\usepackage{minted}
\usepackage{enumitem}
\usepackage[colorlinks,linkcolor=blue]{hyperref}

\usepackage{xeCJK}
\setCJKmainfont{Noto Serif CJK TC}

\renewcommand{\contentsname}{Contents}
\renewcommand{\figurename}{Figure}
\renewcommand{\tablename}{Table}
\hypersetup{
    colorlinks=true,
    linkcolor=black,
    filecolor=magenta,      
    urlcolor=blue,
}
\newcommand{\mytitle}{Digital Image Processing - Homework Assignment\#1}
\newcommand{\myauthor}{B13902022 賴昱錡}

\usepackage{fancyhdr}
\pagestyle{fancy}
\fancyhead{}
\fancyhead[L]{\mytitle}
\fancyhead[R]{\myauthor}

\title{\mytitle}
\author{\textbf{\myauthor}}
\date{Due: 10/13/2025}

\begin{document}

\onehalfspacing
\maketitle

\section{Exercise 1 - Scaling}
\subsection{}
\subsection{}
\subsection{}
\subsection{Bicubic Interpolation}

Bicubic interpolation estimates a value at a 2D grid point using a $4\times4$ neighborhood (16 points) for smoother results. It extends 1D cubic splines separably: 

\begin{enumerate}
\item For each of 4 rows, compute 1D cubic interpolation across 4 columns using offset $t$:
\[
p_k = \sum_{m=0}^{3} c_m(t) \cdot f(i+m-1, j+k-1), \quad k=0\dots3
\]
(e.g., Catmull-Rom basis: $c_0(t) = -0.5t^3 + t^2 - 0.5t$, etc.)

\item Interpolate the 4 $p_k$ vertically using offset $s$:
\[
p(x,y) = \sum_{k=0}^{3} c_k(s) \cdot p_k.
\]
\end{enumerate}

This approximates a degree-3 surface, reducing blur/artifacts in tasks like image scaling.

In bilinear interpolation, we uses $2\times2$ neighborhood (4 points) for linear weighting:
\[
p(x,y) = (1-t)(1-s)f(i,j) + t(1-s)f(i+1,j) + (1-t)sf(i,j+1) + ts f(i+1,j+1).
\]
Separable: horizontal linears, then vertical.

The comparison between the complexity of two methods can be summarized as follows, we can see bicubic trades $\sim$4$\times$ computations for better detail preservation; bilinear prioritizes speed.

\begin{tabular}{|p{2.5cm}|p{5cm}|p{5cm}|}
\hline
\textbf{Aspect} & \textbf{Bilinear} & \textbf{Bicubic} \\
\hline
\textbf{Pixels Used} & 4 (2$\times$2) & 16 (4$\times$4) \\
\hline
\textbf{Operations per Pixel} & $\sim$4 mult + $\sim$2 add & $\sim$20 mult + $\sim$15 add (4$\times$ cubic horiz + 1 vert) \\
\hline
\textbf{Speed} & Very fast ($O(1)$, real-time ok) & 4~5$\times$ slower ($O(1)$, but higher cost) \\
\hline
\textbf{Quality} & Basic, can blur & Sharper, smoother gradients \\
\hline
\end{tabular}

\newpage
\section{Exercise 2 - Distortion}
\subsection{}
\subsubsection*{1. Brown--Conrady Model of Radial Distortion}

The Brown--Conrady model expresses the relation between the ideal (undistorted) image point 
$(x, y)$ and the distorted image point $(x_d, y_d)$ as:

\[
x_d = x \cdot \big( 1 + k_1 r^2 + k_2 r^4 + k_3 r^6 + \dots \big)
\]
\[
y_d = y \cdot \big( 1 + k_1 r^2 + k_2 r^4 + k_3 r^6 + \dots \big)
\]

where:
\begin{itemize}
    \item $(x, y)$ are normalized image coordinates (centered at the principal point),
    \item $r^2 = x^2 + y^2$ is the squared radial distance from the optical axis,
    \item $k_1, k_2, k_3, \dots$ are the radial distortion coefficients.
\end{itemize}

The sign and magnitude of the coefficients determine whether the lens exhibits 
\emph{barrel distortion} or \emph{pincushion distortion}.

\subsubsection*{2. Barrel Distortion}

\begin{itemize}
    \item \textbf{Definition:} Straight lines appear to bulge outwards, like the sides of a barrel.
    \item \textbf{Mathematical explanation:} Occurs when $k_1 < 0$ (dominant case). 
    The scaling factor
    \[
    1 + k_1 r^2 + k_2 r^4 + \dots
    \]
    becomes \emph{smaller} as $r$ increases. 
    Thus, points farther from the image center are mapped closer inward, 
    compressing the edges and making straight lines look convex.
\end{itemize}

Typical example: wide-angle or fisheye lenses.

\subsubsection*{3. Pincushion Distortion}

\begin{itemize}
    \item \textbf{Definition:} Straight lines appear to bend inward, like the edges of a pincushion.
    \item \textbf{Mathematical explanation:} Occurs when $k_1 > 0$ (dominant case).
    The scaling factor
    \[
    1 + k_1 r^2 + k_2 r^4 + \dots
    \]
    grows with $r$. 
    Thus, points farther from the image center are pushed outward, stretching the edges
    and making straight lines bow inward.
\end{itemize}

Typical example: telephoto lenses.

\subsubsection*{4. Visual Summary}

\begin{itemize}
    \item If the radial factor decreases with $r$: \textbf{Barrel distortion} ($k_1 < 0$).
    \item If the radial factor increases with $r$: \textbf{Pincushion distortion} ($k_1 > 0$).
\end{itemize}
\subsection{}
\subsection{}
\subsection{}


\end{document}
% how to display codes?
% \begin{minted}[frame=lines,framesep=2mm,baselinestretch=1.2,linenos,breaklines]{python}
% \end{minted}

% how to display images?
% \begin{figure}[H]
%     \centering
%     \includegraphics[width=0.5\linewidth]{}
%     \caption{Caption}
% \end{figure}
% test
